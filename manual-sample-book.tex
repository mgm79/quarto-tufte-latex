\documentclass[twoside,symmetric]{tufte-book}

\hypersetup{colorlinks}

% Book metadata
\title{A Tufte-Style Book\thanks{Thanks to Edward R.~Tufte for his inspiration.}}
\author[The Tufte-LaTeX Developers]{The Tufte-LaTeX\ Developers}
\publisher{Publisher of This Book}

% Packages
\usepackage{lipsum}
\usepackage{booktabs}
\usepackage{graphicx}
\setkeys{Gin}{width=\linewidth,totalheight=\textheight,keepaspectratio}
\graphicspath{{style-guide/graphics/}}
\usepackage{fancyvrb}
\fvset{fontsize=\normalsize}
\usepackage{xspace}
\usepackage{units}
\usepackage{makeidx}
\makeindex

% Custom commands from original
\newcommand{\hangp}[1]{\makebox[0pt][r]{(}#1\makebox[0pt][l]{)}}
\newcommand{\hangstar}{\makebox[0pt][l]{*}}
\newcommand{\vdqi}{\textit{VDQI}\xspace}
\newcommand{\ei}{\textit{EI}\xspace}
\newcommand{\ve}{\textit{VE}\xspace}
\newcommand{\be}{\textit{BE}\xspace}
\newcommand{\VDQI}{\textit{The Visual Display of Quantitative Information}\xspace}
\newcommand{\EI}{\textit{Envisioning Information}\xspace}
\newcommand{\VE}{\textit{Visual Explanations}\xspace}
\newcommand{\BE}{\textit{Beautiful Evidence}\xspace}
\newcommand{\TL}{Tufte-\LaTeX\xspace}
\newcommand{\monthyear}{%
  \ifcase\month\or January\or February\or March\or April\or May\or June\or
  July\or August\or September\or October\or November\or
  December\fi\space\number\year
}
\newcommand{\openepigraph}[2]{%
  \begin{fullwidth}
  \sffamily\large
  \begin{doublespace}
  \noindent\allcaps{#1}\\% epigraph
  \noindent\allcaps{#2}% author
  \end{doublespace}
  \end{fullwidth}
}
\newcommand{\blankpage}{\newpage\hbox{}\thispagestyle{empty}\newpage}
\newcommand{\measure}[3]{#1/#2$\times$\unit[#3]{pc}}
\newcommand{\na}{\quad--}

\begin{document}

% Front matter
\frontmatter

% Blank page
\blankpage

% Epigraphs page
\newpage\thispagestyle{empty}
\openepigraph{%
The public is more familiar with bad design than good design.
It is, in effect, conditioned to prefer bad design, 
because that is what it lives with. 
The new becomes threatening, the old reassuring.
}{Paul Rand}
\vfill
\openepigraph{%
A designer knows that he has achieved perfection 
not when there is nothing left to add, 
but when there is nothing left to take away.
}{Antoine de Saint-Exup\'{e}ry}
\vfill
\openepigraph{%
\ldots the designer of a new system must not only be the implementor and the first 
large-scale user; the designer should also write the first user manual\ldots 
If I had not participated fully in all these activities, 
literally hundreds of improvements would never have been made, 
because I would never have thought of them or perceived 
why they were important.
}{Donald E. Knuth}

% Title page
\maketitle

% Copyright page
\newpage
\begin{fullwidth}
~\vfill
\thispagestyle{empty}
\setlength{\parindent}{0pt}
\setlength{\parskip}{\baselineskip}
Copyright \copyright\ \the\year\ \thanklessauthor

\par\smallcaps{Published by \thanklesspublisher}

\par\smallcaps{tufte-latex.github.io/tufte-latex/}

\par Licensed under the Apache License, Version 2.0 (the ``License''); you may not
use this file except in compliance with the License. You may obtain a copy
of the License at \url{http://www.apache.org/licenses/LICENSE-2.0}. Unless
required by applicable law or agreed to in writing, software distributed
under the License is distributed on an \smallcaps{``AS IS'' BASIS, WITHOUT
WARRANTIES OR CONDITIONS OF ANY KIND}, either express or implied. See the
License for the specific language governing permissions and limitations
under the License.

\par\textit{First printing, \monthyear}
\end{fullwidth}

% Table of contents
\tableofcontents
\listoffigures
\listoftables

% Dedication
\cleardoublepage
~\vfill
\begin{doublespace}
\noindent\fontsize{18}{22}\selectfont\itshape
\nohyphenation
Dedicated to those who appreciate \LaTeX{} 
and the work of \mbox{Edward R.~Tufte} 
and \mbox{Donald E.~Knuth}.
\end{doublespace}
\vfill
\vfill

% Introduction
\cleardoublepage
\chapter*{Introduction}

This sample book discusses the design of Edward Tufte's books
and the use of the \texttt{tufte-book} and \texttt{tufte-handout} document classes.

% Start main matter
\mainmatter

\chapter{The Design of Tufte's Books}
\label{ch:tufte-design}

\newthought{The pages} of a book are usually divided into three major
sections: the front matter (also called preliminary matter or prelim), the
main matter (the core text of the book), and the back matter (or end
matter).

\newthought{The front matter} of a book refers to all of the material that
comes before the main text.  The following table shows a list of
material that appears in the front matter of \VDQI, \EI, \VE, and \BE
along with its page number.  Page numbers that appear in parentheses refer
to folios that do not have a printed page number (but they are still
counted in the page number sequence).

\bigskip
\begin{minipage}{\textwidth}
\begin{center}
\begin{tabular}{lcccc}
\toprule
 & \multicolumn{4}{c}{Books} \\
\cmidrule(l){2-5} 
Page content & \vdqi & \ei & \ve & \be \\
\midrule
Blank half title page & \hangp{1} & \hangp{1} & \hangp{1} & \hangp{1} \\
Frontispiece\footnotemark{}
  & \hangp{2} & \hangp{2} & \hangp{2} & \hangp{2} \\
Full title page & \hangp{3} & \hangp{3} & \hangp{3} & \hangp{3} \\
Copyright page & \hangp{4} & \hangp{4} & \hangp{4} & \hangp{4} \\
Contents & \hangp{5} & \hangp{5} & \hangp{5} & \hangp{5} \\
Dedication & \hangp{6} & \hangp{7} & \hangp{7} & 7 \\
Epigraph & -- & -- & \hangp{8} & -- \\
Introduction & \hangp{7} & \hangp{9} & \hangp{9} & 9 \\
\bottomrule
\end{tabular}
\end{center}
\end{minipage}
\footnotetext{The contents of this page vary from book to book.  In
  \vdqi this page is blank; in \ei and \ve this page holds a frontispiece;
  and in \be this page contains three epigraphs.}

\section{Typefaces}

Tufte's books primarily use two typefaces: Bembo and Gill Sans.  Bembo is used
for the headings and body text, while Gill Sans is used for the title page and
opening epigraphs in \BE.

Since neither Bembo nor Gill Sans are available in default \LaTeX{}
installations, the \TL document classes default to using Palatino and
Helvetica, respectively.  In addition, the Bera Mono typeface is used for
\texttt{monospaced} type.

\section{Sidenotes}

One of the most prominent and distinctive features of this style is the
extensive use of sidenotes.  There is a wide margin to provide ample room
for sidenotes and small figures.  Any footnotes will automatically
be converted to sidenotes.\footnote{This is a sidenote that was entered
using the footnote command.}  If you'd like to place ancillary
information in the margin without the sidenote mark (the superscript
number), you can use the marginnote command.\marginnote{This is a
margin note.  Notice that there isn't a number preceding the note, and
there is no number in the main text where this note was written.}

\section{Figures and Tables}

Images and graphics play an integral role in Tufte's work.
In addition to the standard figure and tabular environments,
this style provides special figure and table environments for full-width
floats.

\begin{marginfigure}%
  \includegraphics[width=\linewidth]{helix}
  \caption{This is a margin figure.  The helix is defined by 
    $x = \cos(2\pi z)$, $y = \sin(2\pi z)$, and $z = [0, 2.7]$.  The figure was
    drawn using Asymptote.}
  \label{fig:marginfig}
\end{marginfigure}

Figure~\ref{fig:fullfig} is an example of a full-width figure and Figure~\ref{fig:textfig} is an example of the normal
figure environment.

\begin{figure*}[h]
  \includegraphics[width=\linewidth]{sine.pdf}%
  \caption{This graph shows $y = \sin x$ from about $x = [-10, 10]$.
  \emph{Notice that this figure takes up the full page width.}}%
  \label{fig:fullfig}%
\end{figure*}

\begin{figure}
  \includegraphics{hilbertcurves.pdf}
  \caption{Hilbert curves of various degrees $n$. \emph{Notice that this figure only takes up the main textblock width.}}
  \label{fig:textfig}
\end{figure}

Table~\ref{tab:normaltab} shows a table created with the booktabs
package.  Notice the lack of vertical rules---they serve only to clutter
the table's data.

\begin{table}[ht]
  \centering
  \fontfamily{ppl}\selectfont
  \begin{tabular}{ll}
    \toprule
    Margin & Length \\
    \midrule
    Paper width & \unit[8\nicefrac{1}{2}]{inches} \\
    Paper height & \unit[11]{inches} \\
    Textblock width & \unit[6\nicefrac{1}{2}]{inches} \\
    Textblock/sidenote gutter & \unit[\nicefrac{3}{8}]{inches} \\
    Sidenote width & \unit[2]{inches} \\
    \bottomrule
  \end{tabular}
  \caption{Here are the dimensions of the various margins used in the Tufte-handout class.}
  \label{tab:normaltab}
\end{table}

\chapter{On the Use of the tufte-book Document Class}

The \TL document classes define a style similar to the
style Edward Tufte uses in his books and handouts.  Tufte's style is known
for its extensive use of sidenotes, tight integration of graphics with
text, and well-set typography.

\chapter{Typography}

If the Palatino, Helvetica, and Bera Mono typefaces are installed, this style
will use them automatically.  Otherwise, we'll fall back on the Computer Modern
typefaces.

When setting strings of \allcaps{ALL CAPS} or \smallcaps{small caps}, the
letterspacing---that is, the spacing between the letters---should be
increased slightly.  The \texttt{allcaps} command has proper letterspacing for
strings of \allcaps{FULL CAPITAL LETTERS}, and the \texttt{smallcaps} command
has letterspacing for \smallcaps{small capital letters}.  These commands
will also automatically convert the case of the text to upper- or
lowercase, respectively.

\backmatter

\end{document}