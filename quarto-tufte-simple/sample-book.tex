% Options for packages loaded elsewhere
% Options for packages loaded elsewhere
\PassOptionsToPackage{unicode}{hyperref}
\PassOptionsToPackage{hyphens}{url}
\PassOptionsToPackage{dvipsnames,svgnames,x11names}{xcolor}
%
\documentclass[
  letterpaper,
  DIV=11,
  numbers=noendperiod]{scrartcl}
\usepackage{xcolor}
\usepackage{amsmath,amssymb}
\setcounter{secnumdepth}{-\maxdimen} % remove section numbering
\usepackage{iftex}
\ifPDFTeX
  \usepackage[T1]{fontenc}
  \usepackage[utf8]{inputenc}
  \usepackage{textcomp} % provide euro and other symbols
\else % if luatex or xetex
  \usepackage{unicode-math} % this also loads fontspec
  \defaultfontfeatures{Scale=MatchLowercase}
  \defaultfontfeatures[\rmfamily]{Ligatures=TeX,Scale=1}
\fi
\usepackage{lmodern}
\ifPDFTeX\else
  % xetex/luatex font selection
\fi
% Use upquote if available, for straight quotes in verbatim environments
\IfFileExists{upquote.sty}{\usepackage{upquote}}{}
\IfFileExists{microtype.sty}{% use microtype if available
  \usepackage[]{microtype}
  \UseMicrotypeSet[protrusion]{basicmath} % disable protrusion for tt fonts
}{}
\makeatletter
\@ifundefined{KOMAClassName}{% if non-KOMA class
  \IfFileExists{parskip.sty}{%
    \usepackage{parskip}
  }{% else
    \setlength{\parindent}{0pt}
    \setlength{\parskip}{6pt plus 2pt minus 1pt}}
}{% if KOMA class
  \KOMAoptions{parskip=half}}
\makeatother
% Make \paragraph and \subparagraph free-standing
\makeatletter
\ifx\paragraph\undefined\else
  \let\oldparagraph\paragraph
  \renewcommand{\paragraph}{
    \@ifstar
      \xxxParagraphStar
      \xxxParagraphNoStar
  }
  \newcommand{\xxxParagraphStar}[1]{\oldparagraph*{#1}\mbox{}}
  \newcommand{\xxxParagraphNoStar}[1]{\oldparagraph{#1}\mbox{}}
\fi
\ifx\subparagraph\undefined\else
  \let\oldsubparagraph\subparagraph
  \renewcommand{\subparagraph}{
    \@ifstar
      \xxxSubParagraphStar
      \xxxSubParagraphNoStar
  }
  \newcommand{\xxxSubParagraphStar}[1]{\oldsubparagraph*{#1}\mbox{}}
  \newcommand{\xxxSubParagraphNoStar}[1]{\oldsubparagraph{#1}\mbox{}}
\fi
\makeatother


\usepackage{longtable,booktabs,array}
\usepackage{calc} % for calculating minipage widths
% Correct order of tables after \paragraph or \subparagraph
\usepackage{etoolbox}
\makeatletter
\patchcmd\longtable{\par}{\if@noskipsec\mbox{}\fi\par}{}{}
\makeatother
% Allow footnotes in longtable head/foot
\IfFileExists{footnotehyper.sty}{\usepackage{footnotehyper}}{\usepackage{footnote}}
\makesavenoteenv{longtable}
\usepackage{graphicx}
\makeatletter
\newsavebox\pandoc@box
\newcommand*\pandocbounded[1]{% scales image to fit in text height/width
  \sbox\pandoc@box{#1}%
  \Gscale@div\@tempa{\textheight}{\dimexpr\ht\pandoc@box+\dp\pandoc@box\relax}%
  \Gscale@div\@tempb{\linewidth}{\wd\pandoc@box}%
  \ifdim\@tempb\p@<\@tempa\p@\let\@tempa\@tempb\fi% select the smaller of both
  \ifdim\@tempa\p@<\p@\scalebox{\@tempa}{\usebox\pandoc@box}%
  \else\usebox{\pandoc@box}%
  \fi%
}
% Set default figure placement to htbp
\def\fps@figure{htbp}
\makeatother





\setlength{\emergencystretch}{3em} % prevent overfull lines

\providecommand{\tightlist}{%
  \setlength{\itemsep}{0pt}\setlength{\parskip}{0pt}}



 


\KOMAoption{captions}{tableheading}
\makeatletter
\@ifpackageloaded{caption}{}{\usepackage{caption}}
\AtBeginDocument{%
\ifdefined\contentsname
  \renewcommand*\contentsname{Table of contents}
\else
  \newcommand\contentsname{Table of contents}
\fi
\ifdefined\listfigurename
  \renewcommand*\listfigurename{List of Figures}
\else
  \newcommand\listfigurename{List of Figures}
\fi
\ifdefined\listtablename
  \renewcommand*\listtablename{List of Tables}
\else
  \newcommand\listtablename{List of Tables}
\fi
\ifdefined\figurename
  \renewcommand*\figurename{Figure}
\else
  \newcommand\figurename{Figure}
\fi
\ifdefined\tablename
  \renewcommand*\tablename{Table}
\else
  \newcommand\tablename{Table}
\fi
}
\@ifpackageloaded{float}{}{\usepackage{float}}
\floatstyle{ruled}
\@ifundefined{c@chapter}{\newfloat{codelisting}{h}{lop}}{\newfloat{codelisting}{h}{lop}[chapter]}
\floatname{codelisting}{Listing}
\newcommand*\listoflistings{\listof{codelisting}{List of Listings}}
\makeatother
\makeatletter
\makeatother
\makeatletter
\@ifpackageloaded{caption}{}{\usepackage{caption}}
\@ifpackageloaded{subcaption}{}{\usepackage{subcaption}}
\makeatother
\usepackage{bookmark}
\IfFileExists{xurl.sty}{\usepackage{xurl}}{} % add URL line breaks if available
\urlstyle{same}
\hypersetup{
  pdftitle={A Tufte-Style Book},
  pdfauthor={The Tufte-LaTeX Developers},
  colorlinks=true,
  linkcolor={blue},
  filecolor={Maroon},
  citecolor={Blue},
  urlcolor={Blue},
  pdfcreator={LaTeX via pandoc}}


\title{A Tufte-Style Book\thanks{Thanks to Edward
R.\textasciitilde Tufte for his inspiration.}}
\author{The Tufte-LaTeX Developers}
\date{}
\begin{document}
\maketitle

\renewcommand*\contentsname{Table of contents}
{
\hypersetup{linkcolor=}
\setcounter{tocdepth}{3}
\tableofcontents
}
\listoffigures
\listoftables

\chapter{The Design of Tufte's Books}
\label{ch:tufte-design}

\newthought{The pages} of a book are usually divided into three major
sections: the front matter (also called preliminary matter or prelim),
the main matter (the core text of the book), and the back matter (or end
matter).

\newthought{The front matter} of a book refers to all of the material
that comes before the main text. The following table from shows a list
of material that appears in the front matter of \VDQI, \EI, \VE, and \BE
along with its page number. Page numbers that appear in parentheses
refer to folios that do not have a printed page number (but they are
still counted in the page number sequence).

\bigskip
\begin{minipage}{\textwidth}
\begin{center}
\begin{tabular}{lcccc}
\toprule
 & \multicolumn{4}{c}{Books} \\
\cmidrule(l){2-5} 
Page content & \vdqi & \ei & \ve & \be \\
\midrule
Blank half title page & \hangp{1} & \hangp{1} & \hangp{1} & \hangp{1} \\
Frontispiece\footnotemark{}
  & \hangp{2} & \hangp{2} & \hangp{2} & \hangp{2} \\
Full title page & \hangp{3} & \hangp{3} & \hangp{3} & \hangp{3} \\
Copyright page & \hangp{4} & \hangp{4} & \hangp{4} & \hangp{4} \\
Contents & \hangp{5} & \hangp{5} & \hangp{5} & \hangp{5} \\
%Blank & -- & \hangp{6} & \hangp{6} & \hangp{6} \\
Dedication & \hangp{6} & \hangp{7} & \hangp{7} & 7 \\
%Blank & -- & \hangp{8} & -- & \hangp{8} \\
Epigraph & -- & -- & \hangp{8} & -- \\
Introduction & \hangp{7} & \hangp{9} & \hangp{9} & 9 \\
\bottomrule
\end{tabular}
\end{center}
\end{minipage}
\vspace{-7\baselineskip}\footnotetext{The contents of this page vary from book to book.  In
  \vdqi this page is blank; in \ei and \ve this page holds a frontispiece;
  and in \be this page contains three epigraphs.}
\vspace{7\baselineskip}

\bigskip

The design of the front matter in Tufte's books varies slightly from the
traditional design of front matter. First, the pages in front matter are
traditionally numbered with lowercase roman numerals (e.g., i, ii, iii,
iv,\textasciitilde{}\ldots). Second, the front matter page numbering
sequence is usually separate from the main matter page numbering. That
is, the page numbers restart at 1 when the main matter begins. In
contrast, Tufte has enumerated his pages with arabic numerals that share
the same page counting sequence as the main matter.

There are also some variations in design across Tufte's four books. The
page opposite the full title page (labeled ``frontispiece'\,' in the
above table) has different content in each of the books. In \VDQI, this
page is blank; in \EI and \VE, this page holds a frontispiece; and in
\BE, this page contains three epigraphs.

The dedication appears on page\textasciitilde6 in \vdqi (opposite the
introduction), and is placed on its own spread in the other books. In
\ve, an epigraph shares the spread with the opening page of the
introduction.

None of the page numbers (folios) of the front matter are expressed
except in \be, where the folios start to appear on the dedication page.

\newthought{The full title page} of each of the books varies slightly in
design. In all the books, the author's name appears at the top of the
page, the title it set just above the center line, and the publisher is
printed along the bottom margin. Some of the differences are outlined in
the following table.

\bigskip
\begin{center}
\footnotesize
\begin{tabular}{lllll}
\toprule
Feature & \vdqi & \ei & \ve & \be \\
\midrule
Author & & & & \\
\quad Typeface & serif   & serif   & serif   & sans serif \\
\quad Style    & italics & italics & italics & upright, caps \\
\quad Size     & 24 pt   & 20 pt   & 20 pt   & 20 pt \\
\addlinespace
Title & & & & \\
\quad Typeface & serif   & serif   & serif   & sans serif \\
\quad Style    & upright & italics & upright & upright, caps \\
\quad Size     & 36 pt   & 48 pt   & 48 pt   & 36 pt \\
\addlinespace
Subtitle & & & & \\
\quad Typeface & \na     & \na     & serif   & \na \\
\quad Style    & \na     & \na     & upright & \na \\
\quad Size     & \na     & \na     & 20 pt   & \na \\
\addlinespace
Edition & & & & \\
\quad Typeface & sans serif    & \na  & \na  & \na \\
\quad Style    & upright, caps & \na  & \na  & \na \\
\quad Size     & 14 pt         & \na  & \na  & \na \\
\addlinespace
Publisher & & & & \\
\quad Typeface & serif   & serif   & serif   & sans serif \\
\quad Style    & italics & italics & italics & upright, caps \\
\quad Size     & 14 pt   & 14 pt   & 14 pt   & 14 pt \\
\bottomrule
\end{tabular}
\end{center}

\begin{figure*}[p]
\fbox{\includegraphics[width=0.45\linewidth]{graphics/vdqi-title.pdf}}
\hfill
\fbox{\includegraphics[width=0.45\linewidth]{graphics/ei-title.pdf}}
\\\vspace{\baselineskip}
\fbox{\includegraphics[width=0.45\linewidth]{graphics/ve-title.pdf}}
\hfill
\fbox{\includegraphics[width=0.45\linewidth]{graphics/be-title.pdf}}
\end{figure*}

\newthought{The tables of contents} in Tufte's books give us our first
glimpse of the structure of the main matter. \VDQI is split into two
parts, each containing some number of chapters. His other three books
only contain chapters---they're not broken into parts.

\begin{figure*}[p]\index{table of contents}
\fbox{\includegraphics[width=0.45\linewidth]{graphics/vdqi-contents.pdf}}
\hfill
\fbox{\includegraphics[width=0.45\linewidth]{graphics/ei-contents.pdf}}
\\\vspace{\baselineskip}
\fbox{\includegraphics[width=0.45\linewidth]{graphics/ve-contents.pdf}}
\hfill
\fbox{\includegraphics[width=0.45\linewidth]{graphics/be-contents.pdf}}
\end{figure*}

\section{Typefaces}\label{sec:typefaces1}

\index{typefaces} \index{fonts|see{typefaces}}

Tufte's books primarily use two typefaces: Bembo and Gill Sans. Bembo is
used for the headings and body text, while Gill Sans is used for the
title page and opening epigraphs in \BE.

Since neither Bembo nor Gill Sans are available in default \LaTeX{}
installations, the \TL document classes default to using Palatino and
Helvetica, respectively. In addition, the Bera Mono typeface is used for
\texttt{monospaced} type.

The following font sizes are defined by the \TL classes:

\begin{table}[h]\index{typefaces!sizes}
  \footnotesize%
  \begin{center}
    \begin{tabular}{lccl}
      \toprule
      \LaTeX{} size & Font size & Leading & Used for \\
      \midrule
      \verb+\tiny+         &  5 &  6 & sidenote numbers \\
      \verb+\scriptsize+   &  7 &  8 & \na \\
      \verb+\footnotesize+ &  8 & 10 & sidenotes, captions \\
      \verb+\small+        &  9 & 12 & quote, quotation, and verse environments \\
      \verb+\normalsize+   & 10 & 14 & body text \\
      \verb+\large+        & 11 & 15 & \textsc{b}-heads \\
      \verb+\Large+        & 12 & 16 & \textsc{a}-heads, \textsc{toc} entries, author, date \\
      \verb+\LARGE+        & 14 & 18 & handout title \\
      \verb+\huge+         & 20 & 30 & chapter heads \\
      \verb+\Huge+         & 24 & 36 & part titles \\
      \bottomrule
    \end{tabular}
  \end{center}
  \caption{A list of \LaTeX{} font sizes as defined by the \TL document classes.}
  \label{tab:font-sizes}
\end{table}

\section{Headings}\label{sec:headings1}

\index{headings}

Tufte's books include the following heading levels: parts,
chapters,\sidenote{Parts and chapters are defined for the \texttt{tufte-book}
class only.} sections, subsections, and paragraphs. Not defined by
default are: sub-subsections and subparagraphs.

\begin{table}[h]
  \begin{center}
    \footnotesize%
    \begin{tabular}{lcr}
      \toprule
      Heading & Style & Size \\
      \midrule
      Part & roman & \measure{24}{36}{40} \\
      Chapter & italic & \measure{20}{30}{40} \\
      Section & italic & \measure{12}{16}{26} \\
      Subsection & italic & \measure{11}{15}{26} \\
      Paragraph & italic & 10/14 \\
      \bottomrule
    \end{tabular}
  \end{center}
  \caption{Heading styles used in \BE.}
  \label{tab:heading-styles}
\end{table}

\paragraph{Paragraph}

Paragraph headings (as shown here) are introduced by italicized text and
separated from the main paragraph by a bit of space.

\section{Environments}

The following characteristics define the various environments:

\begin{table}[h]
  \begin{center}
    \footnotesize%
    \begin{tabular}{lcl}
      \toprule
      Environment & Font size & Notes \\
      \midrule
      Body text & \measure{10}{14}{26} & \\
      Block quote & \measure{9}{12}{24} & Block indent (left and right) by \unit[1]{pc} \\
      Sidenotes & \measure{8}{10}{12} & Sidenote number is set inline, followed by word space \\
      Captions & \measure{8}{10}{12} &  \\
      \bottomrule
    \end{tabular}
  \end{center}
  \caption{Environment styles used in \BE.}
  \label{tab:environment-styles}
\end{table}

\chapter[On the Use of the tufte-book Document Class]{On the Use of the \texttt{tufte-book} Document Class}
\label{ch:tufte-book}

The \TL document classes define a style similar to the style Edward
Tufte uses in his books and handouts. Tufte's style is known for its
extensive use of sidenotes, tight integration of graphics with text, and
well-set typography. This document aims to be at once a demonstration of
the features of the \TL document classes and a style guide to their use.

\section{Page Layout}\label{sec:page-layout}
\subsection{Headings}\label{sec:headings}

\index{headings} This style provides \textsc{a}- and \textsc{b}-heads
(that is, \Verb|\section| and \Verb|\subsection|), demonstrated above.

If you need more than two levels of section headings, you'll have to
define them yourself at the moment; there are no pre-defined styles for
anything below a \Verb|\subsection|. As Bringhurst points out in
\textit{The Elements of
Typographic Style},\cite{Bringhurst2005} you should ``use as many levels
of headings as you need: no more, and no fewer.'\,'

The \TL classes will emit an error if you try to use
\linebreak\Verb|\subsubsection| and smaller headings.

\% let's start a new thought -- a new section
\newthought{In his later books},\cite{Tufte2006} Tufte starts each
section with a bit of vertical space, a non-indented paragraph, and sets
the first few words of the sentence in \textsc{small caps}. To
accomplish this using this style, use the \doccmddef{newthought}
command:

\begin{docspec}
  \doccmd{newthought}\{In his later books\}, Tufte starts\ldots
\end{docspec}

\section{Sidenotes}\label{sec:sidenotes}

One of the most prominent and distinctive features of this style is the
extensive use of sidenotes. There is a wide margin to provide ample room
for sidenotes and small figures. Any \doccmd{footnote}s will
automatically be converted to
sidenotes.\footnote{This is a sidenote that was entered
using the \texttt{\textbackslash footnote} command.} If you'd like to
place ancillary information in the margin without the sidenote mark (the
superscript number), you can use the \doccmd{marginnote}
command.\marginnote{This is a
margin note.  Notice that there isn't a number preceding the note, and
there is no number in the main text where this note was written.}

The specification of the \doccmddef{sidenote} command is:

\begin{docspec}
  \doccmd{sidenote}[\docopt{number}][\docopt{offset}]\{\docarg{Sidenote text.}\}
\end{docspec}

Both the \docopt{number} and \docopt{offset} arguments are optional. If
you provide a \docopt{number} argument, then that number will be used as
the sidenote number. It will change the number of the current sidenote
only and will not affect the numbering sequence of subsequent sidenotes.

Sometimes a sidenote may run over the top of other text or graphics in
the margin space. If this happens, you can adjust the vertical position
of the sidenote by providing a dimension in the \docopt{offset}
argument. Some examples of valid dimensions are:

\begin{docspec}
  \ttfamily 1.0in \qquad 2.54cm \qquad 254mm \qquad 6\Verb|\baselineskip|
\end{docspec}

If the dimension is positive it will push the sidenote down the page; if
the dimension is negative, it will move the sidenote up the page.

While both the \docopt{number} and \docopt{offset} arguments are
optional, they must be provided in order. To adjust the vertical
position of the sidenote while leaving the sidenote number alone, use
the following syntax:

\begin{docspec}
  \doccmd{sidenote}[][\docopt{offset}]\{\docarg{Sidenote text.}\}
\end{docspec}

The empty brackets tell the \Verb|\sidenote| command to use the default
sidenote number.

If you \emph{only} want to change the sidenote number, however, you may
completely omit the \docopt{offset} argument:

\begin{docspec}
  \doccmd{sidenote}[\docopt{number}]\{\docarg{Sidenote text.}\}
\end{docspec}

The \doccmddef{marginnote} command has a similar \docarg{offset}
argument:

\begin{docspec}
  \doccmd{marginnote}[\docopt{offset}]\{\docarg{Margin note text.}\}
\end{docspec}

\section{References}

References are placed alongside their citations as sidenotes, as well.
This can be accomplished using the normal \doccmddef{cite}
command.\sidenote{The first paragraph of this document includes a citation.}

The complete list of references may also be printed automatically by
using the \doccmddef{bibliography} command. (See the end of this
document for an example.) If you do not want to print a bibliography at
the end of your document, use the \doccmddef{nobibliography} command in
its place.

To enter multiple citations at one
location,\cite[-3\baselineskip]{Tufte2006,Tufte1990} you can provide a
list of keys separated by commas and the same optional vertical offset
argument: \Verb|\cite{Tufte2006,Tufte1990}|.\\

\begin{docspec}
  \doccmd{cite}[\docopt{offset}]\{\docarg{bibkey1,bibkey2,\ldots}\}
\end{docspec}

\section{Figures and Tables}\label{sec:figures-and-tables}

Images and graphics play an integral role in Tufte's work. In addition
to the standard \docenvdef{figure} and \docenvdef{tabular} environments,
this style provides special figure and table environments for full-width
floats.

Full page--width figures and tables may be placed in \docenvdef{figure*}
or \docenvdef{table*} environments. To place figures or tables in the
margin, use the \docenvdef{marginfigure} or \docenvdef{margintable}
environments as follows (see
figure\textasciitilde{}\ref{fig:marginfig}):

\begin{marginfigure}%
  \includegraphics[width=\linewidth]{helix}
  \caption{This is a margin figure.  The helix is defined by 
    $x = \cos(2\pi z)$, $y = \sin(2\pi z)$, and $z = [0, 2.7]$.  The figure was
    drawn using Asymptote (\url{http://asymptote.sf.net/}).}
  \label{fig:marginfig}
\end{marginfigure}

\begin{docspec}
\textbackslash begin\{marginfigure\}\\
  \qquad\textbackslash includegraphics\{helix\}\\
  \qquad\textbackslash caption\{This is a margin figure.\}\\
  \qquad\textbackslash label\{fig:marginfig\}\\
\textbackslash end\{marginfigure\}\\
\end{docspec}

The \docenv{marginfigure} and \docenv{margintable} environments accept
an optional parameter \docopt{offset} that adjusts the vertical position
of the figure or table. See the ``\nameref{sec:sidenotes}'\,' section
above for examples. The specifications are:

\begin{docspec}
  \textbackslash{begin\{marginfigure\}[\docopt{offset}]}\\
  \qquad\ldots\\
  \textbackslash{end\{marginfigure\}}\\
  \mbox{}\\
  \textbackslash{begin\{margintable\}[\docopt{offset}]}\\
  \qquad\ldots\\
  \textbackslash{end\{margintable\}}\\
\end{docspec}

Figure\textasciitilde{}\ref{fig:fullfig} is an example of the
\docenv{figure*} environment and
figure\textasciitilde{}\ref{fig:textfig} is an example of the normal
\docenv{figure} environment.

\begin{figure*}[h]
  \includegraphics[width=\linewidth]{sine.pdf}%
  \caption{This graph shows $y = \sin x$ from about $x = [-10, 10]$.
  \emph{Notice that this figure takes up the full page width.}}%
  \label{fig:fullfig}%
\end{figure*}

\begin{figure}
  \includegraphics{hilbertcurves.pdf}
%  \checkparity This is an \pageparity\ page.%
  \caption[Hilbert curves of various degrees $n$.][6pt]{Hilbert curves of various degrees $n$. \emph{Notice that this figure only takes up the main textblock width.}}
  \label{fig:textfig}
  %\zsavepos{pos:textfig}
\end{figure}

As with sidenotes and marginnotes, a caption may sometimes require
vertical adjustment. The \doccmddef{caption} command now takes a second
optional argument that enables you to do this by providing a dimension
\docopt{offset}. You may specify the caption in any one of the following
forms:

\begin{docspec}
  \doccmd{caption}\{\docarg{long caption}\}\\
  \doccmd{caption}[\docarg{short caption}]\{\docarg{long caption}\}\\
  \doccmd{caption}[][\docopt{offset}]\{\docarg{long caption}\}\\
  \doccmd{caption}[\docarg{short caption}][\docopt{offset}]%
                  \{\docarg{long caption}\}
\end{docspec}

A positive \docopt{offset} will push the caption down the page. The
short caption, if provided, is what appears in the list of
figures/tables, otherwise the
\texttt{long\textquotesingle{}\textquotesingle{}\ caption\ appears\ there.\ Note\ that\ although\ the\ arguments\ \textbackslash{}docopt\{short\ caption\}\ and\ \textbackslash{}docopt\{offset\}\ are\ both\ optional,\ they\ must\ be\ provided\ in\ order.\ Thus,\ to\ specify\ an\ \textbackslash{}docopt\{offset\}\ without\ specifying\ a\ \textbackslash{}docopt\{short\ caption\},\ you\ must\ include\ the\ first\ set\ of\ empty\ brackets\ \textbackslash{}Verb\textbar{}{[}{]}\textbar{},\ which\ tell\ \textbackslash{}doccmd\{caption\}\ to\ use\ the\ default}long'\,'
caption. As an example, the caption to
figure\textasciitilde{}\ref{fig:textfig} above was given in the form

\begin{docspec}
  \doccmd{caption}[Hilbert curves...][6pt]\{Hilbert curves...\}
\end{docspec}

Table\textasciitilde{}\ref{tab:normaltab} shows table created with the
\docpkg{booktabs} package. Notice the lack of vertical rules---they
serve only to clutter the table's data.

\begin{table}[ht]
  \centering
  \fontfamily{ppl}\selectfont
  \begin{tabular}{ll}
    \toprule
    Margin & Length \\
    \midrule
    Paper width & \unit[8\nicefrac{1}{2}]{inches} \\
    Paper height & \unit[11]{inches} \\
    Textblock width & \unit[6\nicefrac{1}{2}]{inches} \\
    Textblock/sidenote gutter & \unit[\nicefrac{3}{8}]{inches} \\
    Sidenote width & \unit[2]{inches} \\
    \bottomrule
  \end{tabular}
  \caption{Here are the dimensions of the various margins used in the Tufte-handout class.}
  \label{tab:normaltab}
  %\zsavepos{pos:normaltab}
\end{table}




\end{document}
